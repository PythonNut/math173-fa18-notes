%&myformat
\documentclass{article}
\usepackage[margin=1in]{geometry}
\usepackage{enumitem}
\usepackage{amsmath}
\usepackage{amssymb}
\usepackage{amsthm}
\usepackage{commath}
\usepackage{dsfont}
\usepackage{xfrac}
\usepackage{siunitx}
\usepackage[parfill]{parskip}
\usepackage[makeroom]{cancel}
\usepackage{mathtools}
\usepackage{tabu}
\usepackage{bm}
\usepackage[utf8]{inputenc}
\usepackage{tikz}
\usepackage{pgfplots}
\usepackage{mdframed}
\usepackage{booktabs}
\usepackage{tikz-cd}
\usepackage{centernot}
% \usepackage{CJKutf8}
% \AtBeginDvi{\input{zhwinfonts}}
% \usepackage{ctex}
\usepackage{xeCJK}
\setCJKmainfont{Source Han Sans}
% \usepackage{physics}
\usepackage{mathrsfs}
\newcommand{\ihat}{\mathbf{\hat i}}
\newcommand{\jhat}{\mathbf{\hat j}}
\newcommand{\khat}{\mathbf{\hat k}}
\newcommand{\rhat}{\mathbf{\hat r}}
\newcommand{\thetahat}{\bm{\hat \theta}}

\DeclareMathOperator{\trace}{tr}
\DeclareMathOperator{\rank}{rank}
\DeclareMathOperator{\range}{range}
\DeclareMathOperator{\nullity}{nullity}
\DeclareMathOperator{\proj}{proj}
\DeclareMathOperator{\vspan}{span}
\DeclareMathOperator{\curl}{curl}
\DeclareMathOperator{\divergence}{div}
\DeclareMathOperator{\hess}{H}
\DeclarePairedDelimiter\ip{\langle }{\rangle}

\renewcommand{\m}[1]{\begin{bmatrix} #1 \end{bmatrix}}
\newcommand{\sm}[1]{\left[\begin{smallmatrix} #1 \end{smallmatrix}\right]}
\newcommand{\vm}[1]{\begin{vmatrix} #1 \end{vmatrix}}
\newcommand{\ZZ}{\mathbb{Z}}
\newcommand{\RR}{\mathbb{R}}
\newcommand{\CC}{\mathbb{C}}
\newcommand{\NN}{\mathbb{N}}
\newcommand{\QQ}{\mathbb{Q}}

\setlength{\parskip}{\bigskipamount}

\renewcommand{\labelenumi}{{\arabic{enumi}.}}
\renewcommand{\labelenumii}{(\roman{enumii})}
\renewcommand{\labelenumiii}{(\arabic{enumiii})}

\newtheorem{definition}{Definition}
\newtheorem{theorem}{Theorem}
\newtheorem{lemma}{Lemma}

% \usepackage{xcolor}
% \definecolor{base03}{HTML}{002b36}
% \definecolor{base0}{HTML}{839496}
% \pagecolor{base03}
% \color{base0}

% \pgfplotsset{compat=1.15}

\makeatletter
\renewcommand*\env@matrix[1][*\c@MaxMatrixCols c]{%
  \hskip -\arraycolsep
  \let\@ifnextchar\new@ifnextchar
  \array{#1}}
\makeatother

\usepackage{xparse}

\ExplSyntaxOn
    \NewDocumentEnvironment{aside} { o }
     {\noindent\ignorespaces\\[0.5em]
       \begin{mdframed}[nobreak=true,backgroundcolor=gray!10]
         \IfNoValueTF{#1}{}{\textbf{#1}}
     }
     {
     \end{mdframed}
     }
\ExplSyntaxOff

\begin{document}

\begin{center}
  \Large Advanced Linear Algebra Week 9 Day 1
  \normalsize

  \textbf{2018/11/12} -- Jonathan Hayase, updated by Prof. Weiqing GU
\end{center}

\section{A Big Picture of Linear ALgebra}

Suppose you have \(A\bm x = \bm b\).
Then let \(\overline A = \sbr{A\mid \bm b}\).
Then there is a solution if and only if \(\rank A = \rank \overline A\).

On the other hand, if \(\det A = 0\) then there is a nontrivial solution to \(A \bm x = \bm 0\).
But if \(\det A \neq 0\) then we have the trivial solution only to \(A\bm x = \bm 0\).

Things like \(\rank\), \(\det\), \(\trace\) are the invariants of \(A\).

Suppose we have a torus and a sphere.
We can always slide a rubber band off a sphere, but that is not so for a torus, so they are not equivalent topologically.

Similarly, if we care about geometry, then an ellipsoid and sphere are not the same.

\subsection{Differential Geometry}

Given a differentiable manifold, we can always find a tangent plane, which lets us push the problem into linear algebra.

Inverse function on manifold: If \(\m{\dif\phi}\) is invertibe, then \(\phi\) is locally invertible.

\subsection{Multivariate Taylor}

\[f(\bm x) = \underbrace{f(\bm x_0) + \nabla f(\bm x_0)(\bm x - \bm x_0)}_{\text{tangent plane}} + \frac{1}{2!}(\bm x - \bm x_0)^T\nabla^2 f(\bm x_0)(\bm x - \bm x_0)\]
If \(\bm x_0\) is a ctirical point, then \(\nabla f(\bm x_0) = \bm 0\).
\begin{align*}
  f(\bm x) - f(\bm x_0) &= \frac{1}{2!}(\bm x - \bm x_0)^T\nabla^2 f(\bm x_0)(\bm x- \bm x_0)\\
                       &= \frac{1}{2!}\ip{\bm x- \bm x_0, \bm x - \bm x_0}_{\nabla^2f(\bm x_0)}
\end{align*}
Where \(\ip{\bm x, \bm y}_M = \bm x^TM\bm y\) as long as \(M\) is positive definite.

\subsection{Current Research Area in Linear Algebra}
Research on approximation algorithms for matrices which cannot fit in memory and/or are sparsly represented.

Topological data analysis can only tell us how many ``holes'' a piece of data has, which is why it ``hasn't gone very far''.

Example: Suppose given \(A \in M^{n \times k}\) and \(B \in M^{k \times n}\) which cannot fit into (your) memory.
But we would still like to find \(AB\).
Suppose we went for some subset \(C\) of \(A\) and \(D\) of \(B\) such that
the result of multiplying the two is still \(n \times m\).

Then suppose our goal was to stochastically achieve \(\norm{AB - CD} < \epsilon\) with probability 99\%.

Prof Gu says: 左行右列

\[\m{a & b\\c &d}\m{1 & 0\\ 0 & 0} = \m{a & 0\\ c & 0}\]
\[\m{a & b\\c &d}\m{0 & 1\\ 1 & 0} = \m{b & d\\ a & c}\]

This means we can do things like
\[\overbrace{\m{\bm v_1 & \bm v_2 & \bm v_3& \bm v_4& \bm v_5}}^A\m{1 & 0 & 0 & \dots\\0 & 0 & 0 & \dots\\ \vdots & \vdots & 1&\dots\\\vdots & \vdots & \vdots&\dots  \\ 0 & 0 & 0 & \dots} = \m{\bm v_1 & \bm v_3}\]

Also want to solve \(A_{n \times k}x_{k \times 1} = b\).
Then we might calculate \(SAx = Sb\), where \(S\) means we will select ``important rows'' of \(A\), such that \(\set{\tilde x^*\mid SAx = Sb}\) and \(\set{x^* \mid Ax = b}\) where \(\tilde x^*\) and \(x^*\) are close to each other with probability 0.99 (or more generally \(1 - \delta\)).

Another problem: If \(A\) is large, then we can't fit into (your) computer's memory.
But we want to find the eigenvalues and vectors of \(A\).

\subsection{Applications}

Suppose you have a UAV flying along a curve, we can attach a frame to every point along the curve representing the local coordinate system of the UAV.

Consider the Lie group (Lie groups are both manifolds and groups)
\[SO(3) = \set{A \mid A^TA = I, \det A = 1}.\]
That is all.

Now, suppose you have a compact manifold.
It's really \(\RR P^3\), the real projective plane, whose double cover is \(S^3\).
However, once we go into the tangent plane, then this becomes linear algebra.
Tangent plane at \(I\) is called the Lie algebra of the Lie group.

Suppose we have a curve \(A(t)\) on the manifold starting at \(I\), then \(A^T(t)A(t) = I\) and \(A(0) = I\).
Suppose we want
\begin{align*}
  \sbr{A^T(t)}' A(t) + A^T(t)A'(t) = 0 && \\
  \sbr{A^T(0)}' A(0) + A^T(0)A'(0) = 0 && \text{plugging in \(t = 0\)}
                                        \sbr{A^T(0)}' + A'(0) = 0\\
  \sbr{A'(0)}^T + A'(0) = 0\\
\end{align*}
So \(A'(0)\) is a skew-symmetric \(3 \times 3\) matrix.
In fact the Lie algebra is the set of skew symmetric matrices.

A Lie algebra is a vector space equipped with the commutator bracket:
\[\sbr{A, B} = AB - BA\]
Satisfying the Jacobi identity:
\[\sbr{\sbr{a, B}, C} + \sbr{\sbr{B, C}, A} + \sbr{\sbr{C,A},B} = 0.\]

\section{Convex optimization}

Suppose \(f\) is twice differentiable.
Then \(f\) is convex if and only if \(\hess f \succ 0\).
\end{document}
